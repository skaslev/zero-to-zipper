\documentclass[pdf]{beamer}
\usepackage{minted}
\setminted{encoding=utf-8}
\usemintedstyle{colorful}
\usepackage{fontspec}
\usepackage{etoolbox}
\AtBeginEnvironment{minted}{\fontsize{8}{8}\selectfont}
\usepackage{amsmath}
\usepackage{amsfonts}
\usepackage{amssymb}
\usepackage{bm}
\usepackage{graphicx}
\usepackage{subfigure}
\usepackage{outlines}
\usepackage{hyperref}
\usepackage{listings}
\mode<presentation>{\usetheme{}}
\title{Zero to Zipper}
%% \subtitle{}
\author{Slavomir Kaslev \\
  \href{mailto:kaslevs@vmware.com}{kaslevs@vmware.com}}

\begin{document}

\begin{frame}
  \titlepage
\end{frame}

\begin{frame}{QOTD}
  \begin{outline}
    \1 ``The shortest path between two truths in the real domain passes through the complex domain.'' \mbox{Jacques Hadamard}
  \end{outline}
\end{frame}

\begin{frame}{The Ocean of Programming Languages}
  \begin{center}
    \includegraphics[scale=0.7]{images/points}
  \end{center}
\end{frame}

\begin{frame}{Comparing Core Languages}
  \begin{table}[]
    \noindent
    \begin{tabular}{|l|l|l|}
      \hline
      \textbf{C}  & \textbf{Haskell}  & \textbf{Lean} \\
      \hline
      data Expr   & data Expr         & data Expr     \\
      data Type   & data Type         &               \\
      data Stmt   &                   &               \\
      \hline
    \end{tabular}
  \end{table}
\end{frame}

\begin{frame}[fragile]{C Core Language 1/3}
  \begin{minted}[escapeinside=~~,mathescape=true]{Haskell}
data Expr
  = Comma       [Expr]
  | Assign      AssignOp Expr Expr
  | Cond        Expr (Maybe Expr) Expr
  | Binary      BinaryOp Expr Expr
  | Cast        Type Expr
  | Unary       UnaryOp Expr
  | SizeofExpr  Expr
  | SizeofType  Type
  | Index       Expr Expr
  | Call        Expr [Expr]
  | Member      Expr Ident Bool
  | Var         Ident
  | Const       Constant
  | CompoundLit Type InitializerList
  \end{minted}
\end{frame}

\begin{frame}[fragile]{C Core Language 2/3}
  \begin{minted}[escapeinside=~~,mathescape=true]{Haskell}
data Type
  = VoidType
  | BoolType
  | CharType
  | IntType
  | FloatType
  | DoubleType
  | ShortType   Type
  | LongType    Type
  | SignedType  Type
  | UnsigType   Type
  | SUType      StructureUnion
  | EnumType    Enumeration
  | FunPtr      Type [Type]
  | Ptr         Type
  | Arr         Type ArraySize
  | TypeDef     Ident
  \end{minted}
\end{frame}

\begin{frame}[fragile]{C Core Language 3/3}
  \begin{minted}[escapeinside=~~,mathescape=true]{Haskell}
data Stmt
  = Label Ident Stmt [Attribute]
  | Case Expr Stmt
  | Default Stmt
  | Expr (Maybe Expr)
  | Compound [Ident] [CompoundBlockItem]
  | If Expr Stmt (Maybe Stmt)
  | Switch Expr Stmt
  | While Expr Stmt Bool
  | For (Either (Maybe Expr) Type)
        (Maybe Expr)
        (Maybe Expr)
        Stmt
  | Goto Ident
  | Cont
  | Break
  | Return (Maybe Expr)
  \end{minted}
\end{frame}

\begin{frame}[fragile]{Haskell Core Language}
  \begin{minted}[escapeinside=~~,mathescape=true]{Haskell}
data Expr b
  = Var   Id
  | Lit   Literal
  | App   (Expr b) (Arg b)
  | Lam   b (Expr b)
  | Let   (Bind b) (Expr b)
  | Case  (Expr b) b Type [Alt b]
  | Tick  (Tickish Id) (Expr b)
  | Type  Type
  | Cast  (Expr b) Coercion
  | Coercion Coercion

data Type
  = TyVarTy   Var
  | LitTy     TyLit
  | AppTy     Type Type
  | ForAllTy  !TyCoVarBinder Type
  | FunTy     Type Type
  | TyConApp  TyCon [KindOrType]
  | CastTy    Type KindCoercion
  | CoercionTy Coercion
  \end{minted}
\end{frame}

\begin{frame}{The Duality of Code and Data}
  \begin{center}
    \includegraphics[scale=0.28]{images/yin-yang}
  \end{center}
\end{frame}

%% \begin{frame}{The Configuration Complexity Clock\footnotemark[1]\hspace*{1pt}}
%%   \begin{center}
%%     \includegraphics[scale=0.50]{images/ConfigurationComplexityClock}
%%   \end{center}
%%   \footnotetext[1]{\url{http://mikehadlow.blogspot.com/2012/05/configuration-complexity-clock.html}}
%% \end{frame}

%% \begin{frame}{The Duality of Algebra and Geometry}
%%   \begin{table}[]
%%   \noindent
%%     \begin{tabular}{ll}
%%       \hline
%%       Algebra     & Geometry    \\
%%       \pause
%%       Left Brain  & Right Brain \\
%%       \pause
%%       Code        & Data        \\
%%       \hline
%%     \end{tabular}
%%   \end{table}
%%   \begin{outline}
%%     \1``‘Should you just be an algebraist or a geometer?’ is like saying ‘Would you rather be deaf or blind?’'' \mbox{Sir Michael Francis Atiyah}
%%     \1``Bad programmers worry about the code. Good programmers worry about data structures and their relationships.'' \mbox{Linus Torvalds}
%%   \end{outline}
%% \end{frame}

\begin{frame}[fragile]{Lean Core Language}
  \begin{minted}[escapeinside=@@,mathescape=true]{Lean}
inductive expr
| var         : nat → expr
| sort        : level → expr
| const       : name → list level → expr
| mvar        : name → name → expr → expr
| local_const : name → name → binder_info → expr → expr
| app         : expr → expr → expr
| lam         : name → binder_info → expr → expr → expr
| pi          : name → binder_info → expr → expr → expr
| elet        : name → expr → expr → expr → expr
| macro       : macro_def → list expr → expr
  \end{minted}
\end{frame}

\begin{frame}{The Curry-Howard Correspondence Extended\footnotemark[4]}
  \begin{center}
    \includegraphics[scale=0.47]{images/hott}
  \end{center}
  \footnotetext[2]{\url{http://saunders.phil.cmu.edu/book/hott-online.pdf\#page=23}}
\end{frame}

\begin{frame}{The Elephant}
  \begin{center}
    \includegraphics[scale=0.28]{images/elephant}
  \end{center}
\end{frame}

\begin{frame}[fragile]{Isomorphisms in Haskell}
  \begin{minted}[escapeinside=~~,mathescape=true]{Haskell}
data Iso a b = Iso { f :: a -> b, g :: b -> a }
-- Should satisfy the following laws:
--   $\forall$ x : a, g (f x) = x
--   $\forall$ x : b, f (g x) = x

inv :: Iso a b -> Iso b a
inv (Iso f g) = Iso g f

comp :: Iso a b -> Iso b c -> Iso a c
comp (Iso f1 g1) (Iso f2 g2) = Iso (f2 . f1) (g1 . g2)
  \end{minted}
\end{frame}

\begin{frame}[fragile]{Isomorphisms in Lean}
  \begin{minted}[escapeinside=@@,mathescape=true]{Lean}
structure iso (a b : Type) :=
(f : a → b) (g : b → a) (gf : Π x, g (f x) = x) (fg : Π x, f (g x) = x)

def inv {a b} (i : iso a b) : iso b a :=
@$\langle$@i.g, i.f, i.fg, i.gf@$\rangle$@

def comp {a b c} (i : iso a b) (j : iso b c) : iso a c :=
@$\langle$@j.f @$\circ$@ i.f, i.g @$\circ$@ j.g, by simp [j.gf, i.gf], by simp [i.fg, j.fg]@$\rangle$@
  \end{minted}
\end{frame}

\begin{frame}{Type Sizes}
  \begin{outline}
    \1 Suppose there's a size function for \textit{finite} types $|\cdot| : Type \to \mathbb{N}$ and let's look at the sizes of the fundamental types namely
    \begin{align*}
      A \oplus B \hspace{1cm} A \otimes B \hspace{1cm} A \to B
    \end{align*}
    \1 One can prove that
    \begin{align*}
      |A \oplus B| &= |A| + |B| \\
      |A \otimes B| &= |A| \cdot |B| \\
      |A \to B| &= |B|^{|A|}
    \end{align*}
  \end{outline}
\end{frame}

\begin{frame}{From Isomorphisms to Equations and Back}
  \begin{outline}
    \1 Let $|A| = a$, $|B| = b$ and $|C| = c$
    \1 Distributive Law
    \begin{align*}
      A \otimes (B \oplus C) &= (A \otimes B) \oplus (A \otimes C) \\
      a(b+c) &= a b + a c
    \end{align*}
    \1 Pattern matching on disjoint union
    \begin{align*}
      A \oplus B \to C &= (A \to C) \otimes (B \to C) \\
      c^{a+b} &= c^{a} c^{b}
    \end{align*}
    \1 Function currying
    \begin{align*}
      A \to B \to C &= A \otimes B \to C \\
      c^{b^{a}} &= c^{a b}
    \end{align*}
  \end{outline}
\end{frame}

\begin{frame}{Analytic Combinatorics}
  \begin{outline}
    \1 Analytic combinatorics deals with counting combinatorial objects by means of their generating functions
    \1 What is a generating function?
    \1 Given a type $A$ and a size function $|\cdot| : A \to \mathbb{N}$, $A$'s ordinary generating function (OGF) is defined as
    \begin{align*}
      A(x) = \sum_{a : A}{x^{|a|}} = \sum_{n=0}^{\infty}{a_n x^n}
    \end{align*}
    \1 The numbers $a_n$ tell us how many objects in $A$ are of size $n$
  \end{outline}
\end{frame}

\begin{frame}[fragile]{Symbolic Method: Finding generating functions}
  \begin{outline}
    \1 Flajolet and Sedgewick propose a simple method of finding equation for the OGF of a given combinatorial construction expressed in their specification language
    \1 In the special case of algebraic data types, the symbolic method uses the fact that if $A, B, C$ are types and $A(x), B(x), C(x)$ are the corresponding OGFs then
    \begin{align*}
      C = A \oplus B &\implies C(x) = A(x) + B(x) \\
    \text{and} \\
      C = A \otimes B &\implies C(x) = A(x) B(x)
    \end{align*}
  \end{outline}
\end{frame}

\begin{frame}[fragile]{Symbolic Method: Examples 1/2}
  \begin{minted}[escapeinside=~~,mathescape=true]{Haskell}
    data Maybe x = None | Just x
  \end{minted}
  \begin{align*}
    M(x) &= 1 + x
  \end{align*}

  \begin{minted}[escapeinside=~~,mathescape=true]{Haskell}
    data List x = Nil | Cons x (List x)
  \end{minted}
  \begin{align*}
    L(x) &= 1 + x L(x)
  \end{align*}
  \begin{align*}
    L(x) &= \frac{1}{1-x}
  \end{align*}
  \begin{align*}
    L(x) &= 1 + x + x^2 + x^3 + x^4 + x^5 + \ldots
  \end{align*}
\end{frame}

\begin{frame}[fragile]{Symbolic Method: Examples 2/2}
  \begin{minted}[escapeinside=~~,mathescape=true]{Haskell}
    data C x = Single x | Pair x x
    type F x = [C x]
  \end{minted}
  \begin{align*}
    F(x) &= \frac{1}{1 - x - x^2}
  \end{align*}
  \begin{align*}
    F(x) &= 1 + x + 2 x^2 + 3 x^3 + 5 x^4 + 8 x^5 + 13 x^6 + 21 x^7 + \ldots
  \end{align*}
  \begin{minted}[escapeinside=~~,mathescape=true]{Haskell}
    data BinTree x = Leaf | Branch x (BinTree x) (BinTree x)
  \end{minted}
  \begin{align*}
    B(x) &= 1 + x B(x)^2 \\
    B(x) &= \frac{1 - \sqrt{1 - 4x}}{2x}
  \end{align*}
  \begin{align*}
    B(x) &= 1 + x + 2x^2 + 5x^3 + 14x^4 + 42x^5 + 132x^6 + 429x^7 + \ldots
  \end{align*}
\end{frame}

\begin{frame}{Zipper}
  \begin{outline}
    \1 Zipper of a data structure is another data structure that provides iteration and modification in $O(1)$ time complexity
    \1 OGF for the zipper over any data structure with OGF $F(x)$ is defined as
    \begin{align*}
      Z_F(x) &= x \frac{\partial}{\partial{x}}F(x)
    \end{align*}
    \1 The derivative of an OGF $\frac{\partial}{\partial{x}}F(x)$ gives the OGF of the same structure with one hole in it, e.g.
    \begin{align*}
      \frac{\partial}{\partial{x}}x^n &= n x^{n-1}
    \end{align*}
    \1 The $x$ in the right hand side is called focus and holds what was initially in that hole
  \end{outline}
\end{frame}

\begin{frame}[fragile]{List Zipper 1/2}
  \begin{minted}[escapeinside=~~,mathescape=true]{Haskell}
    data List x = Nil | Cons x (List x)
  \end{minted}
  \begin{align*}
    L(x) &= 1 + xL(x)
  \end{align*}
  \begin{align*}
    L(x) &= \frac{1}{1-x}
  \end{align*}
  \begin{align*}
    \frac{\partial}{\partial{x}}L(x) &= \frac{1}{(1-x)^2} = L(x)^2
  \end{align*}
  \begin{align*}
    Z_L(x) &= x L(x)^2
  \end{align*}
  \begin{minted}[escapeinside=~~,mathescape=true]{Haskell}
    data ZList x = Focus x (List x) (List x)
  \end{minted}
\end{frame}

\begin{frame}[fragile]{List Zipper 2/2}
  \begin{minted}[escapeinside=~~,mathescape=true]{Haskell}
data ZList a = Focus a [a] [a]

toZipper :: [a] -> Maybe (ZList a)
toZipper [] = Nothing
toZipper (x:xs) = Just $ Focus x xs []

fromZipper :: ZList a -> [a]
fromZipper   (Focus x r []) = x:r
fromZipper z@(Focus x r (y:p)) = fromZipper $ left z

set :: ZList a -> a -> ZList a
set (Focus x r p) y = Focus y r p

left :: ZList a -> ZList a
left z@(Focus x r []) = z
left   (Focus x r (y:p)) = Focus y (x:r) p

right :: ZList a -> ZList a
right z@(Focus x [] p) = z
right   (Focus x (y:r) p) = Focus y r (x:p)

main = do
  let z  = fromJust $ toZipper [1,2,3,4,5]
      z1 = set (right $ right z) 42
      z2 = set (left z1) 0
  print $ fromZipper z2
-- prints [1,0,42,4,5]
  \end{minted}
\end{frame}

\begin{frame}[fragile]{Binary Tree Zipper 1/3}
  \begin{minted}[escapeinside=~~,mathescape=true]{Haskell}
    data BinTree x = Leaf | Branch x (BinTree x) (BinTree x)
  \end{minted}
  \begin{align*}
    B(x) &= 1 + x B(x)^2
  \end{align*}
  \begin{align*}
    \frac{\partial}{\partial{x}}B(x) &= B(x)^2 + 2 x B(x) \frac{\partial}{\partial{x}}B(x)
  \end{align*}
  \begin{align*}
    \frac{\partial}{\partial{x}}B(x) &= \frac{B(x)^2}{1 - 2 x B(x)}
  \end{align*}
  \begin{align*}
    Z_B(x) &= x B(x)^2 \frac{1}{1 - 2 x B(x)}
  \end{align*}
  \begin{minted}[escapeinside=~~,mathescape=true]{Haskell}
    data Segment x = SLeft x (BinTree x) | SRight x (BinTree x)
    data ZBinTree x = Focus x (BinTree x) (BinTree x) [Segment x]
  \end{minted}
\end{frame}

\begin{frame}[fragile]{Binary Tree Zipper 2/3}
  \begin{minted}[escapeinside=~~,mathescape=true]{Haskell}
data BinTree a = Leaf | Branch a (BinTree a) (BinTree a)

data Segment a = SLeft a (BinTree a) | SRight a (BinTree a)
data ZBinTree a = Focus a (BinTree a) (BinTree a) [Segment a]

toZipper :: BinTree a -> Maybe (ZBinTree a)
toZipper Leaf = Nothing
toZipper (Branch x l r) = Just $ Focus x l r []

fromZipper :: ZBinTree a -> BinTree a
fromZipper   (Focus x l r []) = Branch x l r
fromZipper z@(Focus x l r (s:p)) = fromZipper $ up z

set :: ZBinTree a -> a -> ZBinTree a
set (Focus x l r p) y = Focus y l r p

left :: ZBinTree a -> ZBinTree a
left z@(Focus x Leaf r p) = z
left   (Focus x (Branch y ll lr) r p) = Focus y ll lr (SLeft x r:p)

right :: ZBinTree a -> ZBinTree a
right z@(Focus x l Leaf p) = z
right   (Focus x l (Branch y rl rr) p) = Focus y rl rr (SRight x l:p)

up :: ZBinTree a -> ZBinTree a
up z@(Focus x l r []) = z
up   (Focus x l r (SLeft y ur:p))  = Focus y (Branch x l r) ur p
up   (Focus x l r (SRight y ul:p)) = Focus y ul (Branch x l r) p
  \end{minted}
\end{frame}

\begin{frame}[fragile]{Binary Tree Zipper 3/3}
  \begin{minted}[escapeinside=~~,mathescape=true]{Haskell}
t :: BinTree Int
t = Branch 1 (Branch 2 Leaf (Branch 3 Leaf (Branch 4 Leaf Leaf)))
             (Branch 5 Leaf Leaf)

main = do
  let z  = fromJust $ toZipper t
      z1 = set (right $ left z) 42
      z2 = set (up z1) 0
  print $ fromZipper z2

-- prints
-- Branch 1 (Branch 0 Leaf (Branch 42 Leaf (Branch 4 Leaf Leaf)))
--          (Branch 5 Leaf Leaf)
  \end{minted}
\end{frame}

\begin{frame}[fragile]{Challenge}
  \begin{outline}
    \1 Write a library that automatically derives the zipper and its operations for any given data structure
    \vspace{5mm}
    \begin{minted}[escapeinside=~~,mathescape=true]{Haskell}
      data RoseTree a = Node a [RoseTree a]
      $(mkZipper ''RoseTree)
    \end{minted}
    \vspace{3mm}
    \2 Bonus points if it also derives a \lstinline{Comonad} instance
    \vspace{1cm}
    \1 To handle multiple type variables the zipper can be generalized
    \begin{align*}
      Z_F(\bm{x}) &= \bm{x} \cdot \nabla F(\bm{x})
    \end{align*}
  \end{outline}
\end{frame}

\begin{frame}{Further Reading}
  \small
  \begin{outline}
    \1 ``Functional Pearl: The Zipper'' by \mbox{G\'erard Huet}
    \1 ``The Derivative of a Regular Type is its Type of One-Hole Contexts'' by \mbox{Conor McBride}
    \1 ``Analytic Combinatorics'' course on Coursera \url{https://www.coursera.org/learn/analytic-combinatorics}
    \1 ``An Introduction to the Analysis of Algorithms'' by \mbox{Philippe Flajolet} and \mbox{Robert Sedgewick}
    \1 ``Analytic Combinatorics'' by \mbox{Philippe Flajolet} and \mbox{Robert Sedgewick}
    \1 ``Homotopy Type Theory: Univalent Foundations of Mathematics'' by \mbox{The Univalent Foundations Program}
    \1 ``Constructive Mathematics and Computer Programming'' by \mbox{Per Martin-L\"{o}f}
    \1 Tangent bundle \url{https://en.wikipedia.org/wiki/Tangent_bundle}
    \1 Code and slides from this talk \url{https://github.com/skaslev/zero-to-zipper}
  \end{outline}
\end{frame}

\begin{frame}{Thank you}
\end{frame}

\end{document}
